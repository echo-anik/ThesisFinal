% chapters/chapter_2.tex

\section{Preliminaries}
\label{sec:ch2_preliminaries}
To understand the research context, it is essential to define several key concepts. An \textbf{Intrusion Detection System (IDS)} is a security mechanism that monitors network or system activities for malicious activities or policy violations. These systems are broadly classified into two types: \textbf{Signature-based IDS}, which detects threats by looking for specific, known patterns (signatures) of malware; and \textbf{Anomaly-based IDS}, which first establishes a baseline of normal system behavior and then flags any deviation from this baseline as a potential threat. While signature-based systems are effective against known attacks, they are vulnerable to novel, or "zero-day," attacks. Anomaly-based systems, often employing machine learning, can detect novel attacks but face challenges with high false positive rates. This research focuses on the \textbf{Industrial Internet of Things (IIoT)}, an extension of the IoT that connects industrial control systems (ICS) to enterprise networks and the internet. This connectivity exposes critical infrastructure to a wide range of cyber threats, necessitating the development of advanced, anomaly-based IDS tailored to this specific environment.

\section{Review of Existing Research}
\label{sec:ch2_review}
\subsection{Ensemble Learning Approaches}
Ensemble methods, which combine multiple machine learning models to produce a more robust prediction, have consistently proven effective. Early research by Mane et al. demonstrated the high accuracy of tree-based classifiers like Random Forest \cite{mane2023}. A significant challenge in this domain is class imbalance, where attack data is far rarer than normal data. Ahmed et al. addressed this on the UNSW-NB15 dataset by using the Synthetic Minority Oversampling Technique (SMOTE) to significantly improve Random Forest accuracy \cite{ahmed2022}. Further reinforcing the power of ensembles, Mohy-Eddine et al. demonstrated that combining Isolation Forest for outlier removal with Random Forest as the classifier resulted in accuracy scores exceeding 99\% on the Bot-IoT dataset \cite{mohy-eddine2023}. Other works have focused on optimizing the input to these ensembles, with Kasongo using a Genetic Algorithm for feature selection before applying an extreme gradient boosting model to great effect \cite{kasongo2021}.

\subsection{Deep Learning Architectures}
Deep learning has shown remarkable potential for learning the complex, non-linear patterns present in modern network traffic. A broad benchmark by Vinayakumar et al. showed that Deep Neural Networks (DNNs) consistently outperform classical machine learning algorithms across numerous IDS datasets \cite{vinayakumar2019}. A particularly effective hybrid approach, pioneered by researchers like Jiang et al. \cite{jiang2020} and Altunay and Albayrak \cite{altunay2023}, combines Convolutional Neural Networks (CNNs) with Long Short-Term Memory (LSTM) networks. This architecture uses CNNs for spatial feature extraction from packet data and LSTMs to recognize temporal patterns across sequences of network events. This hybrid model proved highly effective on datasets like UNSW-NB15 and X-IIoTID, achieving 93.21\% accuracy for binary classification and 92.9\% for multi-class classification in one study \cite{altunay2023}.

\subsection{Unsupervised and Semi-Supervised Approaches}
Given the difficulty in obtaining large, labeled attack datasets in real-world ICS environments, unsupervised and semi-supervised methods are a critical area of research. Choi and Kim utilized a composite autoencoder, an unsupervised deep learning model, to detect anomalies in the HAI dataset by identifying reconstruction errors \cite{choi2024}. Mahmud et al. specifically proposed the Isolation Forest algorithm as a computationally efficient unsupervised method for the HAI dataset, capitalizing on its ability to isolate anomalies without profiling normal behavior \cite{mahmud2024}. On a different dataset, Long et al. proposed a semi-supervised ladder network with cross-layer connections to improve feature propagation, demonstrating another advanced technique that leverages large amounts of unlabeled data \cite{long2023}.

\section{Summary of Key Findings}
\label{sec:ch2_summary}
The literature review reveals a clear trend towards more sophisticated machine learning models for IDS. While high accuracy is frequently reported, our analysis uncovers several persistent research gaps that form the foundation of this study:
\begin{itemize}
    \item \textbf{Outdated and Imbalanced Datasets:} A primary limitation, as noted in the survey by Rahman et al., is the widespread use of outdated datasets that do not reflect the unique characteristics and attack vectors of modern IIoT environments \cite{rahman2024}. This hinders the generalizability of many proposed solutions.
    \item \textbf{The Accuracy vs. Efficiency Trade-off:} There is a significant gap between computationally intensive deep learning models that achieve high accuracy and the lightweight models required for resource-constrained IoT devices \cite{rahman2024}. A solution that is both highly accurate and operationally efficient is still needed.
    \item \textbf{High False Positive Rates:} A recurring challenge, especially in anomaly-based systems, is the high rate of false positives \cite{wang2023}. This makes it difficult for security operators to distinguish between benign anomalies and actual attacks, reducing the practical utility of the IDS \cite{cao2022}.
    \item \textbf{Need for Multi-Modal Analysis:} Many studies focus on either network traffic data or physical process data in isolation. There is a clear opportunity and need to develop IDS solutions that explicitly fuse these data modalities for more context-aware and robust threat detection \cite{li2018data}.
\end{itemize}
Addressing these gaps—specifically by validating a high-performance, efficient model on a modern, multi-modal dataset like HAI—is the primary focus of this thesis.